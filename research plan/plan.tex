\documentclass[a4paper,11pt]{article}
% \usepackage{fullpage}
\usepackage[top=2.5cm, bottom=2.5cm, left=2.1cm, right=2.1cm]{geometry}
% \usepackage[margin=2.5cm]{geometry}
\usepackage[charter]{mathdesign}
\usepackage{epsfig}
\usepackage{graphicx}
% \usepackage{amssymb}
% \usepackage{amsmath}
\usepackage{verbatim}
\usepackage{fancyhdr}
\usepackage{pifont}% http://ctan.org/pkg/pifont
\usepackage[rgb]{xcolor}
\usepackage{tikz}
\usetikzlibrary{matrix,positioning,fit,shapes,arrows,shadows,calc,backgrounds}
\usepackage{pgfplots}
\usepackage{natbib}
\usepackage{enumitem}
\usepackage{pgfgantt}
\usepackage{booktabs}
\usepackage{url}
\usepackage{xspace}
\usepackage{wrapfig}
\usepackage[show]{chato-notes}
\usepackage{multirow}
\usepackage[official]{eurosym}
\usepackage[bottom]{footmisc}

\usepackage{lipsum}% http://ctan.org/pkg/{graphicx,lipsum}
\newcommand{\PRLsep}{\noindent\makebox[\linewidth]{\resizebox{0.3333\linewidth}{1pt}{$\bullet$}}\bigskip}

\definecolor{verylightmagenta}{rgb}{0.95,0.96,1.0}
\definecolor{brightred}{rgb}{0.90,0.2,0.2}

\newcommand{\ag}[1]{\vspace{1mm}\noindent{\color{orange}{\textbf{Comment:} #1}}}
% \newcommand{\ag}[1]{}
\newcommand{\instructions}[1]{\vspace{1mm}\noindent{\color{blue}{#1}}}
% \newcommand{\instructions}[1]{}

\newcommand{\textbibspace}{{\vspace{1mm}}}

\newcommand{\acronym}{{\sf\small E\ensuremath{x}CLUS}\xspace}
\newcommand{\acronymtitle}{{\sf\large E\ensuremath{x}CLUS}\xspace}
\newcommand{\proposaltitle}{{Advances in explainable clustering}}
\newcommand{\proposalabstitle}{{Advances in explainable clustering}}

%% biblist
\newcommand{\biblist}{\begin{list}{$\bullet$}
  {  \setlength{\itemsep}{2pt}
     \setlength{\parsep}{2pt}
     \setlength{\topsep}{2pt}
     \setlength{\partopsep}{0pt}
     \setlength{\leftmargin}{1.5em}
     \setlength{\labelwidth}{1em}
     \setlength{\labelsep}{1em} } }
\newcommand{\biblistend}{
\end{list}  }

%% biblistnumbered
\newcommand{\biblistn}{\begin{list}{$\bullet$}
  {  \setlength{\itemsep}{2pt}
     \setlength{\parsep}{2pt}
     \setlength{\topsep}{2pt}
     \setlength{\partopsep}{0pt}
     \setlength{\leftmargin}{2em}
     \setlength{\labelwidth}{1em}
     \setlength{\labelsep}{1em} } }
\newcommand{\biblistnend}{
\end{list}  }


% \setcounter{page}{1}

\renewcommand{\baselinestretch}{1.0} 
\begin{document}


\begin{center} 
% {\large Vetenskapsrådet: Distinguished professor grant within natural and engineering sciences 2024} \vspace{2.5mm}\\
{\Large Research plan} \vspace{3mm}\\
{\Large\bf {\proposaltitle} {\sc (}{\acronymtitle}{\sc )}}  \vspace{3mm} \\
{\Large Aristides Gionis} 
\end{center}

\instructions{
The research plan shall be forward-looking and consist of a brief but complete description of the research task. It may cover a maximum of 10 page-numbered A4 pages in Arial, font size 11, single line spacing and 2.5 cm margins, including references and any images.\\
The research plan must include the following headings and information, listed in the following order:
}

\section{Purpose and aims}

\instructions{
State the overall purpose and specific aims of the research project.
}

\section{State of the art}

\instructions{
Summarise briefly the current research frontier within the field or area covered by the project. State key references.
}

\section{Significance and scientific novelty}

\instructions{
Describe briefly how the project relates to previous research within the area, and the impact the project may have in the short and long term. Describe also how the project moves forward or innovates the current research frontier.
}

\section{Preliminary and previous results}

\instructions{
Describe briefly your own previous research and pilot studies within the research area that make it probable that the project will be feasible. If no preliminary results exist, please state this too. State also whether the project contributes further to research and scientific results from a grant awarded previously by the Swedish Research Council.
}

\section{Project description}

\instructions{
Describe the project design under the following headings:
}

\subsection{Theory and methods}

\instructions{
Describe the underlying theory and the methods to be applied in order to reach the project goal.
}

\subsection{Time plan and implementation}

\instructions{
Describe summarily the time plan for the project during the grant period, and how the project will be implemented. Describe also any crucial risks or obstacles that may impact on the implementation, and your plan for managing these.
}

\subsection{Project organization}

\instructions{
Clarify how you and any participating researchers will contribute to the implementation of the project. Explain in particular how the time allocated by you (that is, your activity level) as project leader is suitable for the task, including the relationship with your other research undertakings. Describe and explain the competences and roles of the participating researchers in the project, and also other key persons (including any doctoral students) who are important for the implementation of the project.
}

\section{Equipment}

\instructions{
Describe the basic equipment you and your team have at your disposal for the project.
}

\section{Need for research infrastructure}

\instructions{
Specify the project’s need for international and national research infrastructure. If you choose to use other infrastructure than those supported by the Swedish Research Council External link.and that are thereby open to all, you must justify this (also applies to local research infrastructure).
}

The project is mainly of theoretical nature and will not require extensive computing infrastructure. 
Commodity laptops will be provided to all team members. 
For implementing and evaluating our methods we will use the available 
KTH computing facilities
and the National Academic Infrastructure for Supercomputing in Sweden (NAISS).

\section{International and national collaboration}

\instructions{
Describe your own and the team’s collaboration with foreign and Swedish researchers and research teams. State whether you contribute to or refer to international collaboration in your research.
}

The PI has an extensive international collaboration network. 
Recent and on-going collaborations include
prof.\ De Bie in Ghent University, 
prof.\ Terzi in Boston University,
prof.\ Mannila in Aalto University, and 
Dr.\ Bonchi in Centai Labs.
In spring 2024 the PI will spent one month as a visiting professor 
in Sapienza University of Rome, hosted by prof.\ Leonardi.
In the near future the PI will apply for a sabbatical in Stanford University, 
planning to visit prof.\ Ugander. 
We will seek to strengthen and further expand this collaboration network.
We will encourage the research team to be actively involved in national and international collaborations
and make research visits and internships in other institutes.

\section{Independent line of research}

\instructions{
If you are working or will be working in a larger group, please clarify how your project relates to the other projects in the group. If you are (continuing) working in the same team as your doctoral or postdoc supervisor, or if you are continuing a project that wholly or partly started during your doctoral or postdoc studies, you must also describe the relationship between your project and the research of your former supervisor.
}


\iffalse
{\small
\setlength{\bibsep}{0pt}
\bibliographystyle{abbrv}
\bibliography{references}
}
\fi

% \newpage
% \input{rebound}

\end{document}




